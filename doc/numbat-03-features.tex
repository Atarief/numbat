\section{Features HOWTO}
This section describes various aspects of the Numbat implementation.

\subsection{Finite State Machine}
Numbat does not use FSM framework provided by the Omnet++
environment. Decision to go ahead with own FSM implementation was
based on several factors: 

\begin{itemize}
\item Omnet's FSM are does not allow to stay in the same
  state. Existing workaround to exit and enter the same state is not
  sufficient.
\item There are no clearly defined events in Omnet's FSM.
\item Omnet's FSM must be implemented as object instantiated inside
  another object, which is derived from cSimpleModule.
\item There are not useful timer support.
\end{itemize}

There are several things that must be done in the header (.h) file:
\begin{itemize}
\item Declare class (e.g. WMaxCtrlSS) derived from a Fsm class.
\item Define list of states (as an enum).
\item For each stationary state define onEventState, which will decide
  what to do, when FSM is in a particular state.
\item It is possible to define 2 additional functions: onEnterState
  and onExitState.
\item If the state is transitive (i.e. FSM does not end in this state,
  but rather does some action and then switches to another state
  immediately), onEnterState and onExitState is mandatory.
\item Define list of events (also a enum).
\item (optional) Define timers. Each defined timer can be started,
  when certain conditions are met (e.g. frame is sent and something
  must be done after some time). When timeout is reached, self message
  will be sent. Timers can also be stopped at any time.
\end{itemize}

Below an example of a simple FSM machine implementation is presented. 

\begin{lstlisting}
// WMaxCtrlSs.h file
#include "fsm.h"

class WMaxCtrlSS : public Fsm
{
public:
    WMaxCtrlSS();

    void initialize();
    void handleMessage(cMessage *msg);

protected:
    void fsmInit();

    // --- STATES ---
    typedef enum {
	STATE_OPERATIONAL,         // normal operation
	STATE_SEND_MSHO_REQ,       // send MSHO-REQ message
	STATE_WAIT_BSHO_RSP,       // wait for BSHO-RSP message
	STATE_SEND_HO_IND,         // send HO-IND message
	STATE_HANDOVER_COMPLETE,   // handover complete
	STATE_NUM
    } State;

    // operational state
    static FsmStateType onEventState_Operational(Fsm * fsm, FsmEventType s, cMessage *msg);

    // send MSHO-REQ state
    static FsmStateType onEnterState_SendMshoReq(Fsm *fsm);

    // handover complete state
    static FsmStateType onEventState_HandoverComplete(Fsm * fsm, FsmEventType s, cMessage *msg);

    // --- EVENTS ---
    typedef enum {
	EVENT_HANDOVER_START,
	EVENT_BSHO_RSP_RECEIVED,
	EVENT_NUM
    } Event;

    // --- TIMERS ---
    TIMER_DEF(Handover);
};
\end{lstlisting}

When the class declaration is complete, several things must be
specified in the .cc file:

\begin{itemize}
\item In the fsmInit() method: \verb+statesEventsInit(X, Y, Z)+ must be
  called. X denotes maximum number of states, Y number of events and Z
  is a initial state of the state machine.
\item Initialize all defined states. There are 2 possible state types:
  stationary and transitive.
\item For each stationary state, name and onEventState function must
  be defined. That function will decide what to do when certain event
  has been received.
\item For each transitive state, it is necessary to define name and
  target state. Also \verb+onEnterState+ and \verb+onExitState+
  functions must be defined (those function are optional for
  stationary states).
\item Timers can also be initialized here.
\item handleMessage() method contains a ,,dispatcher''. It translates
  received messages into events, e.g. if received frame is of type
  \verb+WMaxMsgBSHORSP+, then execute
  \verb+onEvent(EVENT_BSHO_RSP_RECEIVED)+.
\item initialize() method should call fsmInit() and also may contain
  additonal initialization code, e.g. timers can be started here.
\item For each stationary state, there should be method called
  onEventState. It defines what exactly should be done when some event
  has occured. That method returns a newState. If no state change is
  necessary, it must contain following entry:
  \verb+return fsm->State();+
\item In the switch() command in the onEventState method, it seems
  reasonable to add \verb+default: CASE_IGNORE(e)+ line, which will
  print appropriate information in case of reception of an event,
  which is not supported in that state.
\item It is possible to defined, what actions should be performed when
  entering (onEnter) or leaving (onExit) particular state.
\item At any moment, \verb+TIMER_START()+ or \verb+TIMER_STOP()+ can
  be executed to start or stop timer.
\end{itemize}

\begin{lstlisting}
Define_Module(WMaxCtrlSS);

WMaxCtrlSS::WMaxCtrlSS()
{

}

void WMaxCtrlSS::fsmInit() {

    // initialize number of states, number of elements and set initial state
    statesEventsInit(WMaxCtrlSS::STATE_NUM, WMaxCtrlSS::EVENT_NUM, STATE_OPERATIONAL);

    // state init
    stateInit(STATE_OPERATIONAL,      "Operational", onEventState_Operational);
    stateInit(STATE_SEND_MSHO_REQ,    "Sending MSHO-REQ", STATE_WAIT_BSHO_RSP, 
                                      onEnterState_SendMshoReq, 0);
    stateInit(STATE_WAIT_BSHO_RSP,    "Waiting for BSHO-RSP", onEventState_WaitForBshoRsp);
    stateInit(STATE_SEND_HO_IND,      "Sending HO-IND", STATE_HANDOVER_COMPLETE, 
                                      onEnterState_SendHoInd, 0);
    stateInit(STATE_HANDOVER_COMPLETE,"Handover complete", onEventState_HandoverComplete);
    stateVerify();

    // event init
    eventInit(EVENT_CDMA_CODE, "CDMA code received");
    eventInit(EVENT_HANDOVER_START, "Begin handover procedure");
    eventInit(EVENT_BSHO_RSP_RECEIVED, "BSHO-RSP received");
    eventVerify();

    TIMER(Handover, 0.1, "Start handover");
}

void WMaxCtrlSS::initialize() {
    // initiate FSM
    fsmInit();

    // Start handover timer
    TIMER_START(Handover);
}

void WMaxCtrlSS::handleMessage(cMessage *msg) 
{
    //IF_TIMER(name, EVENT_TIMEOUT);
    if (msg==TimerHandover) {
	onEvent(EVENT_HANDOVER_START, msg);
	return;
    }

    if (dynamic_cast<WMaxMsgBSHORSP*>(msg)) {
	onEvent(EVENT_BSHO_RSP_RECEIVED, msg);
	return;
    }
}


FsmStateType WMaxCtrlSS::onEventState_Operational(Fsm * fsm, FsmEventType e, cMessage *msg)
{
    switch (e) {
    case EVENT_HANDOVER_START:
	return STATE_SEND_MSHO_REQ;
    default:
	CASE_IGNORE(e);
    }
}

// send MSHO-REQ state
FsmStateType WMaxCtrlSS::onEnterState_SendMshoReq(Fsm *fsm)
{
    WMaxMsgMSHOREQ * mshoReq = new WMaxMsgMSHOREQ("MSHO-REQ");
    mshoReq->setName("MSHO-REQ");
    ev << fsm->fullName() << "Sending MSHO-REQ message." << endl;
    fsm->send(mshoReq, "macOut");
    return fsm->State();
}

// wait for BSHO-RSP state
FsmStateType WMaxCtrlSS::onEventState_WaitForBshoRsp(Fsm * fsm, FsmEventType e, cMessage *msg)
{
    switch (e) {
    case EVENT_BSHO_RSP_RECEIVED:
	return STATE_SEND_HO_IND;
    default:
	CASE_IGNORE(e);
    }
    return fsm->State();
}

// sent HO-IND state
FsmStateType WMaxCtrlSS::onEnterState_SendHoInd(Fsm *fsm)
{
    WMaxMsgHOIND * hoInd = new WMaxMsgHOIND();
    hoInd->setName("HO-IND");
    fsm->send(hoInd, "macOut");
    return fsm->State();
}
    
// handover complete state
FsmStateType WMaxCtrlSS::onEventState_HandoverComplete(Fsm * fsm, FsmEventType s, cMessage *msg)
{
    return fsm->State();
}

\end{lstlisting}
